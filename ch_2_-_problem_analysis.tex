\documentclass[main]{subfiles} 

\begin{document}

\chapter{PROBLEM ANALYSIS}

\section{Understanding the problem}

In the initial days of the project, multiple meetings were held between all the group members to discuss about the topic to work on for the project. After multiple brainstorming sessions, it was decided that the game HeadBall will be made, which is a simple football type game where two characters play against each other in a small field. But this decision gave rise to some further confusions and questions. Some of them are as follows:

\begin{enumerate}
    \item How is it possible to make a simple football type game in C++ using the OOP concepts?
    
    \item What are the different libraries available in C++ for graphics rendering and game development? And which one of them is best suited for our need?
    
    \item Are there libraries available in C++ for physics simulation for games and other projects?
    
    \item How can the game be made fun to play and pretty to look at?
    
    \item How can the GUI aspect be made beautiful and easy to use?
\end{enumerate}
	
\section{Input Requirements}
In the game, there are two players that are to be controlled with a keyboard. They need to be able to move freely and also be able to kick a  ball that is moving in the environment. Similarly, there need to be buttons where the player can click in order to change from one state/screen to the another (like from game state to paused state, i.e. pausing the game).
	
\section{Output Requirements}
The output of the project is a fully functional game where two players can play a simple football game. It should also have a nice menu screen, pause screen where the whole game play pauses, a timer to record the game time and hence declare the half time and full time after certain time passes and also a nice game over screen where the winner of the game is declared. An about screen and instructions screen is also needed for the information about the developers and the information about character control respectively.
	
\section{Processing Requirements}
As the project is developed entirely on 64-bit Linux systems, it is highly recommended to use 64-bit Linux machines or any UNIX based machines (like MacOS). But as the libraries used are cross platform in nature, the users can install the necessary libraries (SFML and Box2D) and build the game themselves without any change in the code in Windows or any other operating system. In such a case, the game works fine without any problems whatsoever.
	
\section{Technical Feasibility}
The project is made feasible by the use of two different open source libraries, SFML and Box2D. SFML is an open source framework that can be used to create 2D graphics. Similarly, Box2D is an open source physics engine that is used to simulate objects to give them real life behaviour. After getting some knowledge about the libraries and also some knowledge about the OOP concepts, the project is feasible to be made by a group of 4 students at a time span of about one and a half months.

\end{document}