\documentclass[main]{subfiles} 

\begin{document}

\chapter{INTRODUCTION}

\pagenumbering{arabic}

This project is based on making a game named \textbf{HeadBall}, which is just like the famous sport football but played by heading the ball. \textbf{HeadBall} is a simple, open-source multiplayer game playable in the same machine where two players take control of two characters. They try to move the players in such a way that they can maneuver a ball resulting the ball to enter the opposition's goal post. They can simply collide with the ball resulting the ball to bounce off or kick the ball which makes the ball move with a greater force. After the time limit is exceeded, the player with the higher number of goals wins. In case the scores tie, the game is declared a draw. The players can also change the character controls as per their requirement and also make other adjustments flexibly in the game as needed. The game is made in C++ with an attempt to apply OOP concepts wherever applicable. Some open source libraries like SFML (for graphics rendering) and Box2D (as a physics engine) are also being used in the project. The engine behind the game was inspired by a YouTube tutorial series: \textit{Flappy Bird SFML Tutorial Series}\cite{sonarsystems_2017_flappy}, where a Flappy Bird clone is made using SFML.

\section{Background and problem statements}
\subsection{Background}
Object Oriented Programming (OOP) is a modern programming paradigm where a complex problem is broken down into smaller and simpler terms known as classes and objects. Using this paradigm, complex applications can be created in a more systematic way by focusing only on a particular part of a problem at once. Similarly the use of other OOP features like inheritance, polymorphism and encapsulation makes the process feel more natural as the classes and objects in the program can be compared with real time objects. Considering the fact, the students of BE I076 were assigned a project to make a real world application making the use of OOP concepts in C++, as a requirement for the completion of course of Object Oriented Programming (CT-451).

Game Development is a highly growing and fascinating field. A fully functional game can be made in any programming language using some programming logic and by making the use of graphics libraries, they can be made attractive and pleasing to the eyes. We also wanted to use the libraries available in C++ and make a fully functioning game out of it. As the process of game development using the existing libraries in C++ was highly based on the OOP paradigm, the game was decided to be made as the project for the completion of course. 

\subsection{Problem Statements}
\begin{enumerate}
    \item How can OOP paradigm be applied to create a real world application?
    
    \item How can different open source libraries be utilized in order to enhance the project?
    
    \item How does game development cycle work and how can a simple and fun game be made using OOP concepts of C++?
    
    \item How can graphics rendering be done in C++ using simple libraries?
    
    \item What are the ways of creating a realistic world in a game?
    
    \item How do animation and sound effects work in game?
    
    \item How can the project be organized and well documented?
    
    \item How can an intuitive and easy to navigate Graphical User Interface (GUI) be implemented?
\end{enumerate}


\section{Objectives}
\begin{enumerate}
    \item To be familiar with basic OOP concepts and its implementation in a real world application.
    
    \item To be familiar about game development and game design.
    
    \item To learn about different open source libraries and also implement those libraries to enhance the project.
    
    \item To make contribution to open source software project.
    
    \item To 1earn about animation of different characters and ways it can be done.
    
    \item To make a fully functional HeadBall game which is fun and entertaining to play and pretty to look at with a beautiful GUI.
    
    \item To gain more experience on group cohesion and teamwork so that it can be applied in a real world scenario.
    
    \item To get familiar with version control with git and collaboration tools like GitHub and VSCode Live Share.
    
    \item To get familiar with industry-standard document typesetting tool such as \LaTeX\ and documentation tool like Doxygen.
\end{enumerate}

\end{document}